\documentclass[rgb,dvipsnames,aspectratio=169,xcolor=table]{beamer}
\usetheme{Copenhagen}

\usepackage[utf8]{inputenc}
\usepackage[T1]{fontenc}
\usepackage{graphicx}
\usepackage{listings}
\lstset{showstringspaces=false}


% Beamer theme settings favored by Troels for this course.  Designed
% to maximise useful space on slides.

\definecolor{main}{HTML}{000000}
\definecolor{fade}{HTML}{cccccc}
\setbeamercolor{structure}{fg=main}
\setbeamercolor{titlelike}{fg=white,bg=main}
\setbeamercolor{background canvas}{bg=white}
\setbeamertemplate{itemize item}{\raisebox{0.8mm}{\rule{1.2mm}{1.2mm}}}
\setbeamercolor{block body}{bg=fade}
\setbeamercolor{block title}{fg=white,bg=main}
\usenavigationsymbolstemplate{} % no navigation buttons

\title{Array Representation}
\author{Troels Henriksen}
\date{22nd of September, 2021}

\begin{document}

\frame{\titlepage}

\begin{frame}
  \frametitle{Arrays}

  \begin{block}{Definition}
    An array is a multidimensional sequence of objects of the
    \textit{same type and size}.
  \end{block}

  \begin{itemize}
  \item Arrays often used to represent mathematical objects such as
    \textit{vectors}, \textit{matrices}, and \textit{tensors}.
  \item Probably the most common data structure for scientific data.
  \item The arrays we will cover in this lab (and course) are
    \begin{itemize}
    \item \textbf{Regular:} all ``rows'' of a multi-dimensional array
      have the same size.
    \item \textbf{Homogeneous:} all elements have the same type.
    \end{itemize}
  \end{itemize}
\end{frame}

\begin{frame}[fragile]
  \frametitle{Regular arrays}

  In Python and F\#, we can have lists of lists with irregular shapes:

\begin{lstlisting}[language=Python]
>>> a = [[1,2,3], [4]]
>>> len(a)
2
>>> len(a[0])
3
>>> len(a[1])
1
\end{lstlisting}

  \begin{itemize}
  \item Such structures are \textbf{irregular}, and outside today's
    topic.
  \item What we will discuss is more similar to NumPy arrays.
  \end{itemize}

\end{frame}

\begin{frame}[fragile]
  \frametitle{So what's wrong with C arrays}

  We can declare an $n\times{}m$ array as

\begin{lstlisting}
double A[n][m];
\end{lstlisting}

And then we can index it with for example \lstinline{A[1][2]}.  Easy!

\pause

\bigskip

But there are many problems with built-in arrays:

\begin{itemize}
\item They decay to pointers in many situations.
\item They cannot be passed to a function without losing their size.
\item They cannot be returned from a function at all.
\item \textbf{They are not values!}
\end{itemize}

\end{frame}

\begin{frame}
  \frametitle{Let's build our own arrays}

  \begin{itemize}
  \item C as a \textit{language} does not have useful dynamic arrays.
  \item But C does have useful support for \textit{dynamic memory
      allocation}.
  \item So let's build our own arrays!
  \end{itemize}

  \begin{block}{Constructing a dynamic array in C}
    \begin{itemize}
    \item Use \texttt{malloc()} or \texttt{calloc()} to obtain a block of
      memory with room for enough \textit{bytes} to fit the array we
      need.
    \item We can view these functions as allocating an ``array of
      bytes'', which we can then \textit{interpret} as arrays of some
      other type.
    \end{itemize}
  \end{block}

  \pause
  \bigskip\textbf{Questions:}

  \begin{itemize}
  \item\textbf{How much memory do we allocate?}  An $x$-element array
    needs \texttt{$x$*sizeof(t)} bytes, where \texttt{t} is the
    element type (\texttt{int}, \texttt{double}, etc).\pause
  \item\textbf{How do we lay out the array in memory?}  That's a more
    open question...
  \end{itemize}

\end{frame}

\begin{frame}[fragile,fragile]
  \frametitle{The idea}

\begin{lstlisting}
int* arr = malloc(12); // reserve 12 bytes
\end{lstlisting}

  \begin{itemize}
  \item Suppose \texttt{malloc()} returns the address $1000$.
  \item When we do \texttt{arr[i]}, C computes the address
    $1000+\texttt{i}\times\texttt{sizeof(int)}$ and reads an
    \texttt{int} (four bytes) from that address.
    \begin{itemize}
    \item \lstinline{&arr[0]}: 1000
    \item \lstinline{&arr[1]}: 1004
    \item \lstinline{&arr[2]}: 1008
    \end{itemize}
  \item (Recall that \lstinline{&x} means ``the address of \texttt{x}''.)
  \end{itemize}

\end{frame}

\begin{frame}[fragile]
  \frametitle{One-dimensional arrays in C}

  \begin{minipage}{0.65\linewidth}
  \lstinputlisting[language=C,basicstyle=\scriptsize]{code/1darray.c}
  \end{minipage}
  \hfill
  \begin{minipage}{0.3\linewidth}
    \scriptsize
\begin{verbatim}
$ gcc 1darray.c -o 1darray
$ ./1darray
&arr: 0x7ffee169ba80
arr:  0x1bb42a0
&arr[0]: 0x1bb42a0 arr[0]: 0
&arr[1]: 0x1bb42a4 arr[1]: 2
&arr[2]: 0x1bb42a8 arr[2]: 4
&arr[3]: 0x1bb42ac arr[3]: 6
&arr[4]: 0x1bb42b0 arr[4]: 8
&arr[5]: 0x1bb42b4 arr[5]: 10
&arr[6]: 0x1bb42b8 arr[6]: 12
&arr[7]: 0x1bb42bc arr[7]: 14
&arr[8]: 0x1bb42c0 arr[8]: 16
&arr[9]: 0x1bb42c4 arr[9]: 18
\end{verbatim}
  \end{minipage}

\end{frame}

\begin{frame}
  \frametitle{Multi-dimensional arrays}

  \begin{itemize}
  \item Machines (and C) provide a \textit{one-dimensional memory (or
      index) space}.
  \item When we want multi-dimensional arrays (and we do!) we need to
    specify a \textit{mapping} between our desired multi-dimensional
    space and the machine's single-dimensional space.
  \end{itemize}

  \begin{center}
    This is an \textit{index function}.
  \end{center}

\end{frame}

\begin{frame}
  \frametitle{Index functions}

\begin{center}
  \textbf{An index function maps a $d$-dimensional index to a
    single-dimensional index.}
\end{center}

\begin{block}{The type of index functions}
\begin{equation*}
  I : \mathbb{N}^{d} \rightarrow \mathbb{N}
\end{equation*}
\end{block}

\begin{center}
  Index functions are not necessarily literal C functions, but a
  \textit{conceptual} description of how the array is laid out in
  memory.
\end{center}

\pause

\begin{block}{Inverse index functions}
\begin{equation*}
  I^{-1} : \mathbb{N} \rightarrow \mathbb{N}^{d}
\end{equation*}
\end{block}
\end{frame}

\begin{frame}

\begin{equation*}
  \begin{pmatrix}
    11 & 12 & 13 & 14 \\
    21 & 22 & 23 & 24 \\
    31 & 32 & 33 & 34
  \end{pmatrix}
\end{equation*}

How do we lay out this matrix in memory?\pause

\bigskip

\begin{description}
\item[\textbf{Row-major order}:] where elements of each \textit{row} are
  contiguous in memory:

  \begin{center}
  \begin{tabular}{|c|c|c|c|c|c|c|c|c|c|c|c|}
    \hline
    11&12&13&14&21&22&23&24&31&32&33&34\\
    \hline
  \end{tabular}
  \end{center}

  with index function
  \[
    (i,j) \mapsto i\times 4 + j
  \]

\pause

\item[\textbf{Column-major order}:] where elements of each \textit{column}
  are contiguous in memory:

  \begin{center}
  \begin{tabular}{|c|c|c|c|c|c|c|c|c|c|c|c|}
    \hline
    11&21&31&12&22&32&13&23&33&14&24&34\\
    \hline
  \end{tabular}
  \end{center}

  with index function
  \[
    (i,j) \mapsto j\times 3 + i
  \]
\end{description}

\end{frame}

\begin{frame}
  \frametitle{Two-dimensional index functions for $n\times{}m$ arrays}

  \begin{minipage}{0.45\textwidth}
    \textbf{Row-major indexing}
    \bigskip
    \begin{equation*}
      (i,j) \mapsto i\times m + j
    \end{equation*}
  \end{minipage}
  \hfill
  \begin{minipage}{0.45\textwidth}
    \textbf{Column-major indexing}
    \bigskip
    \begin{equation*}
      (i,j) \mapsto j\times n + i
    \end{equation*}
  \end{minipage}

  \bigskip

  \textbf{Intuition:}
  \begin{itemize}
  \item Row-major indexing first \textit{skips} $i$ rows
    each comprising $m$ elements, then jumps $j$ elements into the row
    we reach.
  \item This is why $n$ (the number of rows) is not used for row-major
    indexing.
  \end{itemize}
  \bigskip Column-major has same intuition, but we skip size-$n$
  columns instead.

\end{frame}

\begin{frame}[fragile,fragile]
  \frametitle{Two-dimensional arrays in C}

\begin{equation*}
  \begin{pmatrix}
    11 & 12 & 13 & 14 \\
    21 & 22 & 23 & 24 \\
    31 & 32 & 33 & 34
  \end{pmatrix}
\end{equation*}

\begin{minipage}{0.4\linewidth}
\begin{lstlisting}[language=C]
int *A =
  malloc(3*4*sizeof(int));
A[0] = 11; // first row
A[1] = 12; // first row
A[2] = 13; // first row
A[3] = 14; // first row
A[4] = 21; // second row
...
A[11] = 34;
\end{lstlisting}
\end{minipage}
\hfill
\begin{minipage}{0.4\linewidth}
\begin{lstlisting}[language=C]
int *A =
  malloc(3*4*sizeof(int));
A[0] = 11; // first col
A[1] = 21; // first col
A[2] = 31; // first col
A[3] = 12; // second col
A[4] = 22; // second col
...
A[11] = 34;
\end{lstlisting}
\end{minipage}

\end{frame}

\begin{frame}[fragile]
  \frametitle{Index functions as C functions}

\begin{lstlisting}
int idx2_rowmajor(int n, int m, int i, int j) {
  return i * m + j;
}

int idx2_colmajor(int n, int m, int i, int j) {
  return j * n + i;
}
\end{lstlisting}

  \bigskip

  Useful if you get confused when writing index calculations by hand
  (I often do!)

\end{frame}

\begin{frame}[fragile]
  \frametitle{Careful!}

  Consider indexing the $3\times{}4$ array from before with the
  expression

\begin{center}
\lstinline[language=C]{A[idx2_rowmajor(n, m, 2, 5)]}.
\end{center}

\begin{itemize}
\item Trying to access index $(2,5)$---conceptually out of bounds.
\item Index function translates to the flat index $2\times{}3+5=11$.
\item This is \textit{in-bounds} for the 12-element flat array we use
  to represent our matrix!
\item Our program will not crash, but this is probably a bug.
\end{itemize}

\pause

\begin{lstlisting}
int idx2_rowmajor(int n, int m, int i, int j) {
  assert(i >= 0 && i < n);
  assert(j >= 0 && j < m);
  return i * m + j;
}
\end{lstlisting}

\end{frame}

\begin{frame}
  \frametitle{Higher dimensions}

  For a $d$-dimensional row-major array of shape
  $n_{0} \times{} \cdots \times{} n_{d-1}$, the index function where $p$
  is a $d$-dimensional index point is

\begin{equation*}
  p \mapsto \sum_{0 \leq i < d} p_{i}\times \prod_{i<j<d} n_{j}
\end{equation*}

where $p_{i}$ gets the $i$th coordinate of $p$, and the product of an
empty series is $1$.

\bigskip

\textbf{Intuition:} $p_{i}$ tells us how many ``subarrays'' of size
$n_{i+1}...\times...n_{d-1}$ we need to skip.

\end{frame}

\begin{frame}
  \frametitle{Example: four-dimensional indexing}

  Suppose we have an row-major array of shape
  \[
    n_{0} \times n_{1} \times n_{2} \times n_{3}
  \]
  and we wish to compute the flat index of element position
  \[
    (p_{0}, p_{1}, p_{2}, p_{3})
  \]

  We then have to sum these terms where the \textit{strides} $s_{i}$
  depend on the array size:

  \[
    \begin{array}{rl}
      & p_{0} \times \only<1>{s_{0}}\only<2>{n_{1} \times n_{2} \times n_{3}} \\
      + & p_{1} \times \only<1>{s_{1}}\only<2>{n_{2} \times n_{3}} \\
      + & p_{2} \times \only<1>{s_{2}}\only<2>{n_{3}} \\
      + & p_{3} \times \only<1>{s_{3}}\only<2>{1}
    \end{array}
  \]

  \pause

  The stride $s_{i}$ is the product $\prod_{i<j<4} n_{j}$ of the array
  size after dropping the first $i+1$ dimensions.

\end{frame}

\begin{frame}
  \frametitle{Size passing}

  Since we represent arrays as the address of their first element, we
  must manually pass along the size when we call a function with an
  array.

\lstinputlisting[language=C,firstline=1,lastline=7]{code/sumrows.c}

\textbf{As usual:} C will not protect us if we pass the wrong size.
Be careful.

\end{frame}

\begin{frame}
  \frametitle{Slicing}

  \begin{itemize}
  \item When using row-major order, the elements of each row are
    adjacent in memory.
  \item This allows us to perform efficient \textit{slicing}, by
    taking the address of the first element in a row.
  \end{itemize}

\lstinputlisting[language=C,firstline=9,lastline=14]{code/sumrows.c}

\end{frame}

\begin{frame}
  \frametitle{Conclusions}

  \begin{itemize}
  \item C's built-in arrays are suitable only for small arrays,
    typically of static size.
  \item Dynamic allocation can create single-dimensional dynamic
    arrays on the heap.
  \item We can represent multi-dimensional arrays as
    single-dimensional arrays, by specifying an index function.
  \item Careful when indexing these home-made arrays---the C
    language is not much help.
  \end{itemize}

\end{frame}

\end{document}

%%% Local Variables:
%%% mode: latex
%%% TeX-master: t
%%% End:
